\chapter{Installation}\index{Installation}

This chapter provides a guide on how to install WendzelNNTPd 2.x.

\section{Linux/*nix/BSD}

To install WendzelNNTPd from source you need to download the provided archive file (e.g., \emph{wendzelnntpd-2.0.0.tar.gz}) file.\footnote{On some *nix-like operating systems you need to first run \textbf{gzip -d wendzelnntpd-VERSION.tgz} and then \textbf{tar -xf wendzelnntpd-VERSION.tar} instead of letting \texttt{tar} do the whole job.} Extract it and run \textbf{./configure}. Please note that configure indicates missing libraries and packages that you may first need to install using the package system of your operating system.

\begin{verbatim}
$ tar -xzf wendzelnntpd-2.0.0.tgz
$ cd wendzelnntpd
$ ./configure
...
\end{verbatim}

\textbf{Please Note:} \textit{If you wish to compile WITHOUT MySQL or WITHOUT SQlite support}, then run \textbf{MYSQL=NO ./configure} or \textbf{SQLITE=NO ./configure}, respectively.

~

\textbf{Please Note:} \textit{For FreeBSD/OpenBSD/NetBSD/*BSD: There is no MySQL support; you need to use SQLite (i.e., run \texttt{MYSQL=NO ./configure}). Run \texttt{configure} as well as \texttt{make} in the \texttt{bash} shell (under some BSDs you first need to install \texttt{bash}).}

~

\textbf{Please Note:} \textit{If you wish to compile WITHOUT TLS support}, then run \textbf{TLS=NO ./configure}.

~

After \texttt{configure} has finished, run \textbf{make}:

\begin{verbatim}
$ make
...
\end{verbatim}

If you want to generate SSL certificates you can use the helper script:
\begin{verbatim}
	$ sudo ./create_certificate \
		--environment letsencrypt \
		--email <YOUR-EMAIL> \\
		--domain <YOUR-DOMAIN>
\end{verbatim}
For the parameter -{}-environment \textit{local} is also a valid value. Then the certificate is generated only for usage on localhost and is self-signed. After generating the certificate you have to adjust \textit{wendzelnntpd.conf} (check Section \ref{network-settings}) to activate TLS (configuration option \textit{enable-tls})). The paths for certificate and server key can stay as they are. 

~

To install WendzelNNTPd on your system, you need superuser access. Run \textbf{make install} to install it to the default location \textit{/usr/local/*}.

\begin{verbatim}
$ sudo make install
\end{verbatim}

\textbf{Please Note (Upgrades):} Run \textbf{sudo make upgrade} instead of \textbf{sudo make install} for an upgrade. Please cf.\ Section~\ref{Ch:Upgrade} (Upgrading).

\textbf{Please Note (MySQL):} \textit{If you plan to run MySQL}, then no database was set up during 'make install'. Please refer to Section~\ref{Ch:Config} (Basic Configuration) to learn how to generate the MySQL database.

\subsection{Init Script for Automatic Startup}

There is an init script in the directory scripts/startup. It uses the usual parameters like ``start'', ``stop'' and ``restart''.

\section{Unofficial Note: Mac OS X}

A user reported WendzelNNTPd-2.0.0 is installable under Mac OS X 10.5.8. The only necessary change was to add the flag ``\texttt{-arch x86\_64}'' to compile the code on a 64 bit system. However, I never tried to compile WendzelNNTPd on a Mac.

\section{Windows}

Not supported.


