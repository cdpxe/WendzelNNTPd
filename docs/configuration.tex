\chapter{Configuration}\index{Configuration}

This chapter will explain how to configure WendzelNNTPd after installation.

\textbf{Note:} The configuration file for WendzelNNTPd is named \textit{/usr/local/etc/wendzelnntpd.conf}. The format of the configuration file should be self-describing and the default config file includes many comments which will help you to understand the lines you can see and modify there.

\textbf{Note:} On *nix-like operating systems the default installation path is \textit{/usr/local/*} what means that the configuration file of WendzelNNTPd will be \textit{/usr/local/etc/wendzelnntpd.conf}, and the binaries will be placed in \textit{/usr/local/sbin}. %On Windows all files are placed in the directory you have chosen during the installation.

\section{Choosing a database engine}

The first and most important step is to choose a database engine. You can use either SQLite3 (what is the default case and easy to use, but not suitable for larger systems with many thousand postings) or MySQL (which is of course the better solution, but also a little bit more complicated to realize). By default, WendzelNNTPd is configured for SQLite3 and is ready to run. If you want to keep this setting, you do not have to read this section.

\subsection{Modifying wendzelnntpd.conf}

In the configuration file you will find a line called \textbf{database-engine}. You can choose to use either MySQL or SQLite as the backend storage system by appending either \textbf{sqlite} or \textbf{mysql}.

\begin{verbatim}
database-engine mysql
\end{verbatim}

If you choose to use MySQL then you will also need to specify the user and password which WendzelNNTPd should use to connect to the MySQL server. If your server does not run on localhost or uses a non-default MySQL port then you will have to modify these values too.

\begin{verbatim}
; Your database hostname (not needed for sqlite3)
database-server 127.0.0.1

; the database connection port (not needed for sqlite3)
; Comment out to use the default port of your database engine
database-port 3306

; Server authentication (not needed for sqlite3)
database-username mysqluser
database-password supercoolpass
\end{verbatim}

\subsection{Generating your database tables}

Once you have chosen your database backend you will need to create the database and the required tables.

\subsubsection{SQLite}

If you chose SQLite as your database backend then you can skip this step as running \textbf{make install} does this for you.

\textbf{Note:} The SQLite database file as well as the posting management files will be stored in \textit{/var/spool/news/wendzelnntpd/}.

\subsubsection{MySQL}

For MySQL, a SQL script file is included called \textit{mysql\_db\_struct.sql}. It creates the WendzelNNTPd database as well as all the needed tables. Use the MySQL console tool to execute the script.

\begin{verbatim}
$ cd /path/to/your/extracted/wendzelnntpd-archive/
$ mysql -u YOUR-USER -p
Enter password:
Welcome to the MySQL monitor.  Commands end with ; or \g.
Your MySQL connection id is 48
Server version: 5.1.37-1ubuntu5.1 (Ubuntu)

Type 'help;' or '\h' for help. Type '\c' to clear the current input statement.

mysql> source mysql\_db\_struct.sql
...
mysql> quit
Bye
\end{verbatim}

\section{Network Settings}

Now you should specify the IP addresse(s) (IPv4 or IPv6) for WendzelNNTPd to accept connections on and the TCP port (the default NNTP port is 119) to run on.
WendzelNNTPd defines \textbf{conectors} in configuration file.
Each connector defines one port and one listen IP (or all interface IP).
Be careful by using all interface IP(v6) (0.0.0.0 or ::) in connectors.
Within each IP-Version (IPv4 and IPv6) the ports may not overlap between the connectors.
So if you define an all interface IP(v6) within a connector on a dedicated port you must not use any connector with the same port in the same IP-Version.
The connectors are activated top down.

\begin{verbatim}
; You need to specify a connector for each port WendzelNNTPd should
; listen on.

;all interface IPv4
<connector>
   port   119
   listen 0.0.0.0
</connector>

;all interface IPv6
<connector>
   port   119
   listen ::
</connector>
\end{verbatim}

\section{Setting the Allowed Size of Postings}

To change the maximum size of a post to be sent to the server, change the variable \textbf{max-size-of-postings}. The value must be set in Bytes and the default value is 20971520 (20 MBytes).

\begin{verbatim}
max-size-of-postings 20971520
\end{verbatim}

\section{Verbose Mode}

If you have any problems running WendzelNNTPd or if you simply want more information about what is happening, you can uncomment the \textbf{verbose-mode} line.

\begin{verbatim}
; Uncomment 'verbose-mode' if you want to find errors or if you
; have problems with the logging subsystem. All log strings are
; written to stderr too, if verbose-mode is set. Additionally all
; commands sent by clients are written to stderr too (but not to
; logfile)
verbose-mode
\end{verbatim}

\section{Security Settings}

WendzelNNTPd supports TLS Version 1.0 to 1.3.
TLS settings can be defined for each connector independently.
To place the certificates following folders have been created in the configuration directory by installation script:
\begin{verbatim}
   ssl.ca          # CA certificates for client certificate validation
   ssl.crl         # CRL of CA certificates
   ssl.crt         # Server Certificates
   ssl.key         # Server Keys (not public readable)   
\end{verbatim}

Following options can be defined in each connector:
\begin{verbatim}
; You need to specify a connector for each port WendzelNNTPd should
; listen on.
<connector>
;    Initially enable TLS on connector - denies session without TLS
   enable-ssl
;    Enable STARTTLS in connection
   enable-starttls
;    Set available ciphers for TLS1.0 to 1.2 - see cipher string from OpenSSL
   openssl-ciphers "ALL:!COMPLEMENTOFDEFAULT:!eNULL"
;    Set available cipher suites for TLS1.3
  openssl-cipher-suites "TLS_AES_128_GCM_SHA256:TLS_AES_256_GCM_SHA384"
;    Set available TLS versions
   openssl-tlsversion "1.0-1.3"
;    Server certificate
   openssl-servercertificate "/usr/local/etc/ssl.crt/server.crt"
;    Key for Server certificate
   openssl-serverkey "/usr/local/etc/ssl.key/server.key"
;    Force server cipher order
   openssl-server-cipher-preference
;    Request client authentication by certificate
;    openssl-verifyclient [ none | optional | require ]
   openssl-verifyclient optional
;    Specify nuber of CA levels to verify eg. 0==selfsignes Certs only
   openssl-verifydepth 5
;    Path for CA certificates used to verify client certificates
;    Use c_rehash <Path> to build required hashes for certificates!
   openssl-CApath "/usr/local/etc/ssl.ca"
;    File that holds all CA certificates used to verify client certificates
   openssl-CAfile "/usr/local/etc/ssl.ca/cacert.pem"
;    Use first CN in Subject of client certificate to authenticate NNTP user
   openssl-CNauthentication
;    Enable CRL-checking on client certificates
;    openssl-CRLcheck [ none | leaf | chain ]
   openssl-CRLcheck leaf
;    Specify path that holds CRL vertificates
;    Use c_rehash <Path> to build required hashes for certificates!
   openssl-CRLpath "/usr/local/etc/ssl.crl"
;    File that holds all CRL certificates used for checking
   openssl-CRLfile "/usr/local/etc/ssl.crl/cacert.crl"
;    Set port to listen on
   port    119
;    Set IPv4 od IPv6 to use with connector
   listen  0.0.0.0
;</connector>
\end{verbatim}

\textbf{enable-ssl} enables TLS on the connector.
TLS has to be negotiated on session setup.
No sessions without TLS are permitted.
At least \textbf{openssl-servercertificate} and \textbf{openssl-serverkey} must be specified.
\textbf{enable-ssl} supersedes \textbf{enable-starttls} if defined in same connector.

~

\textbf{enable-starttls} activates STARTTLS on connector.
TLS can be negotiated during session with \textit{STARTTLS} command. 
Encrypted and unencrypted sessions are supported on the same connector.
At least \textbf{openssl-servercertificate} and \textbf{openssl-serverkey} must be specified.

~

\textbf{openssl-ciphers} defines the Configuration-String for ciphers in TLS1.0 to TLS1.2.
A detailed description of the string can be found in man pages of \textit{openssl-ciphers}.

~

\textbf{openssl-cipher-suites} defines the active Cipher-Suites for TLS1.3.

~

\textbf{openssl-tlsversion} defines possible TLS-versions to be negotiated by the client.
The option must be in the following format: \texttt{openssl-tlsversion 1.[min]-1.[max]}

~

\textbf{openssl-servercertificate} defines the file with the server certificate in PEM format.

~

\textbf{openssl-serverkey} defines the associated private key for the server certificate in PEM format.

~

\textbf{openssl-server-cipher-preference} uses the server defined order of ciphers in TLS negotiation.

~

\textbf{openssl-verifyclient} defines the authentication method for the clients.
Possible values are:
\begin{itemize}
	\item{none} - Certificate from the client is not requested and not needed. This is the \textbf{Default}.
	\item{optional} - Certificate from the client is requested but not mandatory.
	\item{require} - Certificate from the client is requested an mandatory.
\end{itemize}
Time values of the certificates are checked on validation.

~

\textbf{openssl-verifydepth} defines the length of the certificate chain that is searched in the trust store for client certificate verification. A value of 0 means that each client certificate needs to be in the trust store for successful authentication.

~

\textbf{openssl-CApath} directory where trusted certificates are stored. Each file contains one certificate in PEM format.

~

\textbf{openssl-CAfile} file where trusted certificates are stored. The file contains a sequence of certificates in PEM format.

~

\textbf{openssl-CRLcheck} defines how CRL-check on client certificates is performed.
Possible values are:
\begin{itemize}
	\item{none} - No CRL-check is performed. This is the \textbf{Default}.
	\item{leaf} - CRL check is performed for client certificate.
	\item{chain} - CRL check is performed on whole chain of client certificate.
\end{itemize}

~

\textbf{openssl-CRLpath} directory where CRL-certificates are stored. Each file contains one CRL-certificate in PEM format.

~

\textbf{openssl-CRLfile} file where CRL-certificates are stored. The file contains a sequence of CRL-certificates in PEM format.

~

Certificates in \textbf{openssl-CApath} and \textbf{openssl-CRLpath} must have symbolic links named with their hash values. 
Use \textit{c\_rehash $<$dir$>$} shipped with OpenSSL to create the links.

~

\textbf{openssl-CNauthentication} uses the first CommonName (CN) in subject to authenticate the client. If CN is found in usertable, the session is put to authenticated state.
\textbf{use-authentication} should be activated in configuration file.

~

\textbf{listen} defines the IP(v6) address to listen on. You can use any-IP(v6) address (0.0.0.0 or ::) to listen on all interfaces of the IP-version. Be aware that the listeners must not overlap within their IP-version.

~

\textbf{port} defines the port to listen on. If the port is below 1023 the the program must be run as root.

\subsection{Authentication and Access Control Lists (ACL)}

WendzelNNTPd contains an extensive access control subsystem. If you want to only allow authenticated users access to the server, you should uncomment \textbf{use-authentication}. This gives every authenticated user access to each newsgroup.

\begin{verbatim}
; Activate authentication
use-authentication
\end{verbatim}

If you need a slightly more advanced authentication system, you can activate Access Control Lists (ACL) by uncommenting \textbf{use-acl}. This activates the support for Role-based ACL too.

\begin{verbatim}
; If you activated authentication, you can also activate access
; control lists (ACL)
use-acl
\end{verbatim}

\subsection{Anonymized Message-ID}

By default, WendzelNNTPd makes a user's hostname or IP address part of new message-IDs when a user sends a post using the NNTP POST command. If you do not want that, you can force WendzelNNTPd not to do so by uncommenting \textbf{enable-anonym-mids}, which enables anonymized Message-IDs.

\begin{verbatim}
; This prevents that IPs or Hostnames will become part of the
; message ID generated by WendzelNNTPd what is the default case.
; Uncomment it to enable this feature.
enable-anonym-mids
\end{verbatim}

\subsection{Changing the Default Salt for Password Hashing}

When uncommenting the keyword \textbf{hash-salt}, the default salt value that is used to enrich the password hashes can be changed. Please note that you have to define the salt \textit{before} you set-up the first password since it will otherwise be stored as hashed with an old salt, rendering it unusable. For this reason, it is best to define your salt right after running \textbf{make install}.

\begin{verbatim}
; This keyword defines a salt to be used in conjunction with the
; passwords to calculate the cryptographic hashes. The salt must
; be in the form [a-zA-Z0-9.:\/-_]+.
; ATTENTION: If you change the salt after passwords have been
; stored, they will be rendered invalid! If you comment out
; hash-salt, then the default hash salt defined in the source
; code will be used.
hash-salt 0.hG4//3baA-::_\
\end{verbatim}

WendzelNNTPd applies the SHA-2 hash algorithm using a 256 bit hash value. Please also note that the final hash is calculated using a string that combines salt, username and password as an input to prevent password-identification attacks when an equal password is used by multiple users. However, utilizing the username is less secure than having a completely separate salt for every password.


